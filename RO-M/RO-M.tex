\documentclass[answers, a4paper, 11pt]{exam}
\usepackage{amsmath}
\usepackage{amssymb}
\usepackage{amsthm}
\usepackage[italian]{babel}
\usepackage{ccicons}
\usepackage{hyperref}
\usepackage{cleveref}
\usepackage[utf8]{inputenc}
\usepackage[autostyle=false, style=english]{csquotes}
\usepackage[margin=2cm]{geometry}
\usepackage{graphicx}
\usepackage{mathrsfs}
\usepackage{multicol}
\usepackage{relsize}
\usepackage{parskip}
\pagestyle{plain}
\graphicspath{{./images/}}
\MakeOuterQuote{"}
\setlength{\columnseprule}{.4pt}
\renewcommand{\solutiontitle}{\noindent\textbf{R:}\enspace}
\def\dbar{{\mathchar'26\mkern-12mu d}}
\title{Ricerca Operativa M}
\author{Kevin Michael Frick}
\begin{document}
\maketitle
\begin{questions}
	\question Programmazione matematica
	\begin{parts}
\part Relativamente all'affermazione 'ogni punto di un politopo è combinazione convessa dei vertici'
\begin{solution}

\end{solution}

\part Dato un politopo P definito dai vincoli di un LP, condizione necessaria e sufficiente perché un punto sia un vertice è che
\begin{solution}

\end{solution}

\part Cosa è una base di una matrice A?
\begin{solution}

\end{solution}
\part Quale tra le seguenti è la definizione di politopo?
\begin{solution}

\end{solution}

\part Sia P un politopo, H un generico iperpiano, HS uno dei due semispazi generati da H. Insieme dei punti f = intersezione tra P e HS è detta:
\begin{solution}

\end{solution}


\end{parts}

\question Programmazione lineare


\part Una SBA si dice degenere se
\begin{solution}

\end{solution}

\end{parts}
	\question Algoritmo del simplesso
	\begin{parts}
\part In un tableau del simplesso primale, gli elementi della colonna 0 righe da 1 a m
\begin{solution}

\end{solution}
\part Quale di queste non è un'assunzione che l'algoritmo del Simplesso deve verificare (o comunque sapere che sia valida) prima di poter operare?
\begin{solution}

\end{solution}

\part Cosa afferma la regola di Dantzig?
\begin{solution}

\end{solution}

\part In cosa consiste la Fase 1 dell'algoritmo del Simplesso?
\begin{solution}

\end{solution}

\part Cosa significa se nel calcolo di theta max c'è un caso di parità?
\begin{solution}

\end{solution}

\part Se $y_{ij}$ è minore o uguale a 0 per ogni i, in relazione a theta, cosa succede?
\begin{solution}

\end{solution}

\part Quale colonna conviene far entrare in base in un cambiamento di base?
\begin{solution}

\end{solution}

\part Cosa contiene un tableau nella posizione di riga 0 e colonna 0?
\begin{solution}

\end{solution}

\part Cosa dice il criterio di ottimalità?
\begin{solution}

\end{solution}

\part Cosa significa se la soluzione del problema artificiale ha valore positivo?
\begin{solution}

\end{solution}

\part Se la soluzione del problema artificiale della fase 1 del simplesso ha valore nullo
\begin{solution}

\end{solution}

\part Cosa contiene il tableau a qualunque iterazione?
\begin{solution}

\end{solution}



	\end{parts}

\question Dualità
	\begin{parts}
\part Relativamente al simplesso duale, quale tra le seguenti affermazioni è errata?
\begin{solution}

\end{solution}

\part Il lemma di Farkas
\begin{solution}

\end{solution}

\part Nell'algoritmo del simplesso duale
\begin{solution}

\end{solution}
\part In un tableau del simplesso duale, gli elementi della riga 0 (colonna da 1 a n)
\begin{solution}

\end{solution}

\part Quale può essere una possibile coppia di problemi primale-duale?
\begin{solution}

\end{solution}

\part Se un problema di programmazione lineare (primale) ha soluzione ottima finita, allora:
\begin{solution}

\end{solution}

\part Quale tra queste affermazioni è falsa rispetto ad una corrispondenza primale-duale?
\begin{solution}

\end{solution}

	\end{parts}

	\question Programmazione lineare intera
	\begin{parts}
\part Aggiungendo un taglio di Gomory al tableau finale di un LP con yi0 non intero
\begin{solution}

\end{solution}

\part Se A è totalmente unimodulare, relativamente a problemi ILP
\begin{solution}

\end{solution}
\part Nel Forward Step della strategia di esplorazione Depth-First rivisitata
\begin{solution}

\end{solution}

\part Relativamente ad un problema ILP e il suo rilassamento continuo LP
\begin{solution}

\end{solution}


\part Dopo aver inserito i vincoli del procedimento Branch-and-Bound nel tableau
\begin{solution}

\end{solution}

\part L'algoritmo Knapsack DP è
\begin{solution}

\end{solution}
\part Si dice che S' domina S'' se
\begin{solution}

\end{solution}


	\end{parts}

	\question Complessità
	\begin{parts}

\part Relativamente alla classe di problemi NP, quale di queste affermazioni è errata?
\begin{solution}

\end{solution}


\part Cosa è un'istanza di un problema?
\begin{solution}
\end{solution}

\part Relativamente al prezzo ombra
\begin{solution}

\end{solution}

\part Il teorema degli scarti complementari afferma
\begin{solution}

\end{solution}

\part Cosa succede se, dopo aver individuato la riga con elemento in colonna 0 negativo, nell'algoritmo del simplesso duale ogni elemento di quella riga è positivo o nullo?
\begin{solution}
\end{solution}

\part Relativamente ad un problema fortemente NP-Completo
\begin{solution}
\end{solution}

\part Cosa è la dimensione di un problema?
\begin{solution}
\end{solution}


\part Quale di queste affermazioni è errata?
\begin{solution}
\end{solution}

\part Una matrice m x n è totalmente unimodulare se
\begin{solution}
\end{solution}

\part Data una funzione f convessa su un insieme S convesso, la corda che unisce due punti della funzione
\begin{solution}

\end{solution}

\part In un problema di programmazione lineare con m vincoli ed n variabili, le condizioni di ortogonalità (complementary slackness)
\begin{solution}

\end{solution}

\part La situazione: primale illimitato e corrispondente duale illimitato
\begin{solution}

\end{solution}

\part Dati un vettore di parametri a ed un vettore di incognite x, una disequazione della forma a'x<=b è equivalente a
\begin{solution}

\end{solution}

\part Sia v il vertice del politopo corrispondente alla base attuale B. Se B è degenere, un'operazione di pivoting del simplesso primale sposta la soluzione ad un vertice, che è
\begin{solution}

\end{solution}

\part Se due basi producono la stessa soluzione base ammissibile x, allora x contiene
\begin{solution}

\end{solution}

\part Se un tableau del simplesso primale corrisponde alla soluzione ottima
\begin{solution}

\end{solution}

\part Nell'algoritmo del simplesso duale, sia a'i (i>0) una riga corrispondente ad un valore negativo in colonna 0. Se tutti i coefficienti della a'i sono positivi o nulli, ciò implica che
\begin{solution}

\end{solution}

\part Nell'operazione di pivoting del simplesso primale, in caso di parità nella scelta del pivot, la nuova soluzione base
\begin{solution}

\end{solution}

\part In un tableau del simplesso primale, gli elementi della riga 0 (colonne da 1 a n)
\begin{solution}

\end{solution}

\part Nel metodo delle due fasi, se al termina della fase 1 la soluzione ha valore negativo
\begin{solution}

\end{solution}

\part L'aggiunta al tableau del taglio di Gomory relativo ad una riga generatrice frazionaria produce una soluzione (un tableau) che
\begin{solution}

\end{solution}

\part Un insieme di m colonne di una matrice intera A m x n è linearmente indipendente se
\begin{solution}

\end{solution}

\part La combinazione convessa stretta di due punti distinti x ed y di un politopo convesso è
\begin{solution}

\end{solution}

\part La programmazione lineare
\begin{solution}

\end{solution}

\part Una variabile libera in segno può essere sostituita equivalentemente da
\begin{solution}

\end{solution}

\part Quando un primale è illimitato, la situazione: corrispondente duale impossibile
\begin{solution}

\end{solution}

\part Dato il politopo definito dai vincoli di un LP, una combinazione convessa di vertici ottimi è
\begin{solution}

\end{solution}

\part Sia v il vertice del politopo corrispondente alla base attuale B. Se B non è degenere, un'operazione di pivoting del simplesso primale sposta la soluzione ad un vertice che è
\begin{solution}

\end{solution}

\part Dati una funzione convessa definita su un insieme S convesso ed una soglia t, il sottoinsieme di punti x di S in cui f(x) <= t
\begin{solution}

\end{solution}

\part Nel metodo delle due fasi, se al termina della fase 1 la soluzione ha valore positivo
\begin{solution}

\end{solution}

\part Un insieme di colonne di una amtrice intera A m x n non è linearmente indipendente se
\begin{solution}

\end{solution}

\part In un tableau del simplesso duale, i costi relativi si trovano
\begin{solution}

\end{solution}

\part La scelta del pivot del simplesso duale viene determinata da
\begin{solution}

\end{solution}

\part Una soluzione base corrispondente ad una sottomatrice base B di una matrice A m x n si ottienne
\begin{solution}

\end{solution}

\part Sia P il problema ILP e L(P) il suo rilassamento continuo. Se L(P) è illimitato, allora
\begin{solution}

\end{solution}

\part Le condizioni di ortogonalità (complementary slackness) di una coppia primale-duale garantiscono
\begin{solution}

\end{solution}

\part Nell'algoritmo del simplesso primale, utilizzare ad ogni iterazione la regola di Dantzig
\begin{solution}

\end{solution}

\part In un algoritmo branch-and-bound per un problema di massimizzazione, sia U l'upper-bound del nodo corrente. Il nodo viene ucciso se
\begin{solution}

\end{solution}

\part Un algoritmo polinomiale per un problema NP-completo
\begin{solution}

\end{solution}


\end{parts}


\textbf{Disclaimer}:  Questo documento può contenere errori e imprecisioni che potrebbero danneggiare sistemi informatici, terminare relazioni e rapporti di lavoro, liberare le vesciche dei gatti sulla moquette e causare un conflitto termonucleare globale.
Procedere con cautela.
Questo documento è rilasciato sotto licenza CC-BY-SA 4.0. \ccbysa
\end{document}

